\usepackage[utf8]{inputenc}
\usepackage{amsmath}
\usepackage{amsfonts}
\usepackage{amssymb}
\usepackage{makeidx}
\usepackage{graphicx}
\usepackage{layout}
\usepackage{lscape}
\usepackage{xfrac}
\usepackage{tabularx}
\usepackage{rotating}
\usepackage{longtable}
%\usepackage{todonotes}
\usepackage{textcomp, xspace}

\usepackage{physics}


\usepackage{hyperref}
\hypersetup{
  colorlinks=true,
  linktoc=all,
  citecolor=black,
  linkcolor=black,
  urlcolor=red,
  % linktocpage=true,
}
\usepackage{cleveref}

\usepackage{listings}

\usepackage[square,numbers,sort&compress, sectionbib]{natbib}
\usepackage{chapterbib}

\setlength\bibhang{.3in}

\usepackage{tocloft} % for table of contents

\usepackage{lineno}
\usepackage{setspace}
\onehalfspacing
\usepackage{microtype}
\usepackage{color}
\usepackage{fancyhdr}
\usepackage[labelfont=bf]{caption}
% For electronic submission use:
\usepackage[inner=2cm,outer=2cm,top=2cm=bottom=2cm]{geometry}
% For printing use:
%\usepackage[inner=4cm,outer=2cm,top=2cm=bottom=2cm]{geometry}
\usepackage{tocbibind}

\usepackage{subfig}    % needed for \subfloat
\usepackage{titlesec}
\usepackage{derivative}
\usepackage{enumerate} % needed for the enumerate environment

\usepackage{subfiles} % Best loaded last in the preamble

%glossary for acronyms
\usepackage[acronym,nonumberlist,toc,section=subsection,numberedsection=nolabel]{glossaries}
\newacronym{QiML}{QiML}{Quantum-Inspired Machine Learning}
\newacronym{QML}{QML}{Quantum Machine Learning}
\newacronym{QNN}{QNN}{Quantum Neural Network}
\newacronym{VQC}{VQC}{variational quantum circuit}
\newacronym{QAOA}{QAOA}{Quantum Approximate Optimization Algorithm}
\newacronym{CNOT}{CNOT}{controlled-NOT}

\newacronym{BERT}{BERT}{Bidirectional Encoder Representations from Transformers}
\newacronym{GPT}{GPT}{Generative Pre-trained Transformer}

\newacronym{JIT}{JIT}{Just-In-Time}

% \newacronym{AUC}{AUC}{area under the curve}

\newacronym{UWA}{UWA}{the University of Western Australia}
\newacronym{HPC}{HPC}{High-Performance Computing}
\newacronym{IRDS}{IRDS}{Institutional Research Data Store}


\setlength{\headheight}{14.5pt}
\addtolength{\topmargin}{-2.5pt}

\fancyhf{}
\fancyhead[L]{\bfseries\nouppercase{\leftmark\hfill\rightmark}}
\fancyfoot[C]{\thepage}
\pagestyle{fancy}

\newcommand\frontmatter{\pagenumbering{roman}}
\newcommand\mainmatter{\cleardoublepage\pagenumbering{arabic}}
